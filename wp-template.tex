% \documentclass[12pt]{extarticle}
\documentclass[12pt]{scrartcl}
\RequirePackage{tabularx}
\RequirePackage{indentfirst}
\RequirePackage{polyglossia}
\RequirePackage{graphicx}
\RequirePackage[dvipsnames]{xcolor}
\usepackage{multirow}
\usepackage{tabularray}
\setmainlanguage{russian}
\setotherlanguage{english}
\setkeys{russian}{babelshorthands=true} % Переносы типа
                                        % красно"=зеленый
\RequirePackage{hyperref}
\hypersetup{
    colorlinks=true,
    linkcolor=black,
    filecolor=black,
    urlcolor=black,
    pdftitle={Рабочая программа курса ИМИТ ИГУ},
    % pdfpagemode=FullScreen,
    }
% \RequirePackage{minted}
%
\RequirePackage{luatextra}
\RequirePackage{unicode-math}
\defaultfontfeatures{Scale=MatchLowercase,Numbers=Lining,Ligatures=TeX}
\RequirePackage{microtype}
\SetProtrusion
    [name=std]
    {
      encoding={utf8},
      family=*}
    {
    « = {300,     },
    » = {    , 300},
    „ = {300,     },
    “ = {    , 300},
    ( = {300,     },
    ) = {    , 300},
    ! = {    , 300},
    ? = {    , 300},
    : = {    , 300},
    ; = {    , 300},
    . = {    , 300},
    - = {    , 300},
   {,}= {    , 300}
 }
 \microtypesetup{protrusion=true,expansion=true}

\setmainfont{Times New Roman}
\setsansfont{Fira Mono}
\setmonofont{Fira Code}[
     Scale=MatchLowercase,
     Numbers=SlashedZero,
     Ligatures=TeX,
     Numbers=Lining]
\newfontfamily{\cyrillicfonttt}{Fira Code}[
     Scale=MatchLowercase,
     Numbers=SlashedZero,
     Ligatures=TeX,
     Numbers=Lining]
\newfontfamily{\cyrillicfont}{Times New Roman}[
     Scale=MatchLowercase,
     Numbers=SlashedZero,
     Ligatures=TeX,
     Numbers=Lining]
\newfontfamily{\cyrillicfontsf}{Fira Mono}[
     Scale=MatchLowercase,
     Numbers=SlashedZero,
     Ligatures=TeX,
     Numbers=Lining]


     \RedeclareSectionCommand[%
     font=\large\rmfamily\bfseries
     ]{section}
     \RedeclareSectionCommand[%
     font=\rmfamily\bfseries
     ]{subsection}
     \RedeclareSectionCommand[%
     font=\normalsize\rmfamily\bfseries
     ]{subsubsection}
     \RedeclareSectionCommand[%
     font=\normalsize\rmfamily\bfseries
     ]{paragraph}
     \DeclareTOCStyleEntry{dottedtocline}{section}

\makeatletter
% \renewcommand\sectionlinesformat[4]{%
%   \@hangfrom{\hskip #2#3}{\MakeUppercase{#4}}%
% }
\makeatother
%\renewcommand{\sectioncatchphraseformat}[4]{%
%  \hskip #2#3\MakeUppercase{#4}%
%}
\graphicspath{ {./files} }

     \setlength{\parindent}{1cm}
     \setcounter{tocdepth}{\subsectiontocdepth}

     \newcommand{\rdf}[2]{#2}
     \newenvironment{rdfctx}[1]{}{}
     \renewcommand{\paragraph}[1]{\par\textbf{#1}}

{{#defcontext}}
{{/defcontext}}

\rdf{%
  \rdfsetctx{syll}{wpdb:_}%
%  \rdfsetctx{common}{wpdb:_}%
}{}

\begin{document}
\begin{titlepage}
  \begin{center}
    \includegraphics{isu.png}\par
  {\bfseries МИНИСТЕРСТВО ОБРАЗОВАНИЯ И НАУКИ РОССИЙСКОЙ ФЕДЕРАЦИИ}\par
  {\bfseries федеральное государственное бюджетное образовательное учреждение}\par
  {\bfseries высшего образования}\par
  {\bfseries \MakeUppercase{ <<{{context.university rdfs:label}}>>} }\par
{ФГБОУ ВО «{{context.university.abbrev}}»}\par
% \MakeUppercase{ {{context.institute rdfs:label}} }\par
\vspace{1ex}
{\bfseries Кафедра \MakeLowercase{ {{disc idd:chair rdfs:label}}} }
\end{center}
\vspace{2em}
\begin{tabularx}{\textwidth}{XX}
  & УТВЕРЖДАЮ \\
  & \\
  & {{context.institute.position}} {{context.institute.abbrev}} {{context.university.abbrev}} \\
  & \\
  & \underline{\hspace{4.5cm}}\;{{context.director foaf:name}}\\
  & \\
  & <<\underline{\hspace{1cm}}>>\underline{\hspace{5cm}}~2022~г.
\end{tabularx}
\vfill
\begin{center}\large
  {\bfseries \MakeUppercase{Рабочая программа дисциплины (модуля)} }\\[1ex]
  % {\bfseries {{disc dcterms:identifier}}~{{disc idd:discipline rdfs:label}}}
\end{center}
\vfill
\begin{tblr}{X[4,l]X[5,l]}
  {\bfseries Наименование дисциплины (модуля):} & {{disc dcterms:identifier}}~{{disc idd:discipline rdfs:label}} \\
  {\bfseries Направление подготовки:} & {\hyphenpenalty=100000 {{curr idd:specialty dcterms:identifier}}~{{context.specialty rdfs:label}}}\\
  {\bfseries Направленность (профиль) подготовки:} & {{context.profile rdfs:label}}\\
  {\bfseries Квалификация выпускника:} & {{context.level rdfs:label}}\\[1em]
  {\bfseries Форма обучения:} & {{context.mural rdfs:label}}
\end{tblr}
\vfill
\vfill
\begin{center}
  {{#with context}}
  % {{city}}\;--\;{{enrolledIn}}
  {{city}}~{{enrolledIn}}~г.
  {{/with}}
\end{center}
\end{titlepage}
\newpage
\tableofcontents
\newpage
\section{Цели и задачи дисциплины}
\begin{rdfctx}{\rdfsetctx{dc}{syll wpdd:courseDC !wpdd:CourseDC}}
\paragraph{Цели:}\rdf{dc wpdd:aim !cnt:ContentAsText}{....}
\paragraph{Задачи:}\rdf{dc wpdd:problem !cnt:ContentAsText}{....}
\end{rdfctx}

\section{Место дисциплины в структуре ОПОП ВО}
Учебная дисциплина {{disc dcterms:identifier}}~<<{{disc idd:discipline rdfs:label}}>> относится к
Блоку~{{disc idd:block dcterms:identifier}} образовательной программы, {{disc idd:blockPart rdfs:comment}}.

\begin{rdfctx}{\rdfsetctx{dc}{syll wpdd:courseDC !wpdd:CourseDC}}
Для изучения данной учебной дисциплины необходимы знания, умения и навыки,
формируемые предшествующими дисциплинами: \rdf{dc wpdd:require !cnt:ContentAsText}{%
  \begin{enumerate}
  \item <<....>>
  \end{enumerate}}

Перечень последующих учебных дисциплин, для которых необходимы знания, умения и
навыки, формируемые данной учебной дисциплиной: \rdf{dc wpdd:ensure !cnt:ContentAsText}{%
  \begin{enumerate}
  \item <<....>>
  \end{enumerate}}
\end{rdfctx}


\section{Требования к результатам освоения дисциплины}

Процесс освоения дисциплины направлен на формирование следующих компетенций в
соответствии с ФГОС ВО и ОП ВО по направлению подготовки {{#with context.specialty}}
{{dcterms:identifier}}~{{rdfs:label}}: \par {{/with}}
{{#rdfeach disc idd:hasCompetence}}

\noindent {{this dcterms:identifier}}\ --\ <<{{this rdfs:label}}>> \par

{{/rdfeach}}

\begin{rdfctx}{\rdfsetctx{dc}{syll wpdd:courseDC !wpdd:CourseDC}}
В результате освоения дисциплины обучающийся должен

\paragraph{Знать:} \rdf{dc wpdd:know !cnt:ContentAsText}{...}

\paragraph{Уметь:} \rdf{dc wpdd:ableTo !cnt:ContentAsText}{...}

\paragraph{Владеть:} \rdf{dc wpdd:posess !cnt:ContentAsText}{...}
\end{rdfctx}

\section{Содержание и структура дисциплины}

Объем дисциплины составляет {{disc idd:credits idd:expert}} зачетных ед., {{disc idd:hours idd:expert}} час.

Форма промежуточной аттестации: {{disc idd:controlType rdfs:label}}.

\subsection{Содержание дисциплины, структурированное по темам, c указанием видов
  учебных занятий и отведенного на них количества академических часов} % 4.1
{{#rdfeach disc idd:term }}
% --------------------------------------------------
% Семестр {{this idd:number }}
% Курс {{this idd:course}}
% Контроль (зачет-8, экзамен-10) {{this idd:contol}}
% Зачетных единиц - {{this idd:credits}}
% Лекции (часы) - {{this idd:lection}}
% Практических работ (часы) - {{this idd:practice}}
% Лабораторных работ (часы) - {{this idd:laboratoryWorks}}
% Самостоятельная работа (часы) - {{this idd:independentWork}}
% Подготовка к зачету/экзамену (часы) - {{this idd:preparation}}
% Всего часов в СЕМЕСТРЕ - {{this idd:total}}
% Форма контроля {{this idd:type rdfs:label}}


\noindent{\footnotesize%
%  \begin{tblr}{|p{5cm}|m{1cm}|m{1cm}|m{1cm}|m{1cm}|m{1cm}|p{2cm}|}
  \begin{tblr}{|X[4,l]|X[1,c]|X[1,c]|X[1,c]|X[1,c]|X[1,c]|X[2,l]|}
  \hline
  \SetCell[r=3]{c} Раздел дисциплины~/ тема &
  \SetCell[r=3]{c} Семес. &
  \SetCell[c=4]{c} Виды учебной работы & & & &
  \SetCell[r=3]{l} Формы текущего контроля; Формы промежут. аттестации \\\hline
  & &  \SetCell[c=3]{l,4cm} Контактная работа преподавателя с обучающимися & & &
  \SetCell[r=2]{c} Самост. работа & \\ \hline
  & &  Лекции & Лаб. занятия & Практ. занятия & & \\\hline
{{#rdflet this idd:number = idd-number }}
  {{#rdfeach syll wpdd:itemList ?wpdd:TopicList ?wpdd:ItemList}}
  {{#with this ^schema:member ?wpdd:ListItem ?wpdd:Topic}}
  Тема~{{this dcterms:identifier}}.~{{this rdfs:label}} & {{idd-number}} & & & & & \\\hline
  {{/with}}
  {{/rdfeach}}
{{/rdflet}}
            Итого ({{this idd:number}} семестр) & &  {{this idd:lection}} &  {{this idd:laboratoryWorks}} &
 {{this idd:practice}} & {{this idd:independentWork}} & {{this idd:type rdfs:label}} \\
 \hline
\end{tblr}}


{{/rdfeach}}

\subsection{План внеаудиторной самостоятельной работы обучающихся по дисциплине} % 4.2

{{#rdfeach disc idd:term }}
% --------------------------------------------------
% Семестр {{this idd:number }}
% Курс {{this idd:course}}
% Контроль (зачет-8, экзамен-10) {{this idd:contol}}
% Зачетных единиц - {{this idd:credits}}
% Лекции (часы) - {{this idd:lection}}
% Практических работ (часы) - {{this idd:practice}}
% Лабораторных работ (часы) - {{this idd:laboratoryWorks}}
% Самостоятельная работа (часы) - {{this idd:independentWork}}
% Подготовка к зачету/экзамену (часы) - {{this idd:preparation}}
% Всего часов в СЕМЕСТРЕ - {{this idd:total}}
% Форма контроля {{this idd:type rdfs:label}}


\noindent{\footnotesize%
%  \begin{tblr}{|p{5cm}|m{1cm}|m{1cm}|m{1cm}|m{1cm}|m{1cm}|p{2cm}|}
  \begin{tblr}{|X[4,l]|X[1,l]|X[1,c]|X[1,c]|X[2,l]|X[1.5,l]|}
  \hline
  \SetCell[r=2]{c} Раздел дисциплины~/ тема &
  \SetCell[c=3]{c,10em} Самостоятельная работа обучающихся & & &
  \SetCell[r=2]{c,5em} Оценочное средство &
  \SetCell[r=2]{l,5em} Учебно-методическое обеспечение самост. работы \\\hline
  & Вид самост. работы & Сроки выполнения & Затраты времени &  & \\\hline
  {{#rdfeach syll wpdd:itemList ?wpdd:TopicList ?wpdd:ItemList}}
  {{#with this ^schema:member ?wpdd:ListItem ?wpdd:Topic}}
  Тема~{{this dcterms:identifier}}.~{{this rdfs:label}} &  & & & & \\\hline
  {{/with}}
  {{/rdfeach}}
  \SetCell[c=3]{l,22em} Общая трудоемкость самостоятельной работы (час.) & & &
  {{this idd:independentWork}} & &\\\hline
  \SetCell[c=3]{l,25em} {Из них с использованием электронного обучения\\ и дистанционных образовательных} технологий (час.) & & & & &\\
 \hline
\end{tblr}}


{{/rdfeach}}


\subsection{Содержание учебного материала} % 4.3

\begin{rdfctx}{\rdfsetctx{list}{syll wpdd:itemList !wpdd:TopicList !wpdd:ItemList}}
  \begin{rdfctx}{\rdfsetctx{item}{list ^schema:member !wpdd:ListItem !wpdd:Topic}}
    Тема~\rdf{item dcterms:identifier}{1}.~\rdf{item rdfs:label}{Название темы.}
    \rdf{item dcterms:description}{Подтема...., подтема ....}
  \end{rdfctx}


\end{rdfctx}

\subsubsection{Перечень семинарских, практических занятий и лабораторных работ} % 4.3.1

{{#rdfeach disc idd:term }}
% --------------------------------------------------
% Семестр {{this idd:number }}
% Курс {{this idd:course}}
% Контроль (зачет-8, экзамен-10) {{this idd:contol}}
% Зачетных единиц - {{this idd:credits}}
% Лекции (часы) - {{this idd:lection}}
% Практических работ (часы) - {{this idd:practice}}
% Лабораторных работ (часы) - {{this idd:laboratoryWorks}}
% Самостоятельная работа (часы) - {{this idd:independentWork}}
% Подготовка к зачету/экзамену (часы) - {{this idd:preparation}}
% Всего часов в СЕМЕСТРЕ - {{this idd:total}}
% Форма контроля {{this idd:type rdfs:label}}


\noindent{\footnotesize%
%  \begin{tblr}{|p{5cm}|m{1cm}|m{1cm}|m{1cm}|m{1cm}|m{1cm}|p{2cm}|}
  \begin{tblr}{|X[4,l]|X[1,c]|X[2,l]|X[1,c]|}
  \hline
  Тема занятия &
  Всего часов & Оценочные средства & Формируемые компетенции \\\hline
  {{#rdfeach syll wpdd:itemList ?wpdd:PracticeTopicList ?wpdd:ItemList}}
  {{#with this ^schema:member ?wpdd:ListItem ?wpdd:PracticeTopic}}
  Тема~{{this dcterms:identifier}}.~{{this rdfs:label}} &  & & \\\hline
  {{/with}}
  {{/rdfeach}}
\end{tblr}}


{{/rdfeach}}


\subsubsection{Перечень тем (вопросов), выносимых на самостоятельное изучение студентами
в рамках самостоятельной работы} % 4.3.2

{{#rdfeach disc idd:term }}
% --------------------------------------------------
% Семестр {{this idd:number }}
% Курс {{this idd:course}}
% Контроль (зачет-8, экзамен-10) {{this idd:contol}}
% Зачетных единиц - {{this idd:credits}}
% Лекции (часы) - {{this idd:lection}}
% Практических работ (часы) - {{this idd:practice}}
% Лабораторных работ (часы) - {{this idd:laboratoryWorks}}
% Самостоятельная работа (часы) - {{this idd:independentWork}}
% Подготовка к зачету/экзамену (часы) - {{this idd:preparation}}
% Всего часов в СЕМЕСТРЕ - {{this idd:total}}
% Форма контроля {{this idd:type rdfs:label}}


\noindent{\footnotesize%
%  \begin{tblr}{|p{5cm}|m{1cm}|m{1cm}|m{1cm}|m{1cm}|m{1cm}|p{2cm}|}
  \begin{tblr}{|X[3,l]|X[3,l]|X[1,c]|}
  \hline
  Тема &
  Задание & Формируемые компетенции \\\hline
  {{#rdfeach syll wpdd:itemList ?wpdd:IndependentWorkTopicList ?wpdd:ItemList}}
  {{#with this ^schema:member ?wpdd:ListItem ?wpdd:IndependentWorkTopic}}
  Тема~{{this dcterms:identifier}}.~{{this rdfs:label}} &  & \\\hline
  {{/with}}
  {{/rdfeach}}
\end{tblr}}


{{/rdfeach}}



\subsection{Методические указания по организации самостоятельной работы студентов} % 4.4

Самостоятельная работа студентов всех форм и видов обучения является одним из
обязательных видов образовательной деятельности, обеспечивающей реализацию
требований Федеральных государственных стандартов высшего образования. Согласно
требованиям нормативных документов самостоятельная работа студентов является
обязательным компонентом образовательного процесса, так как она обеспечивает
закрепление получаемых на лекционных занятиях знаний путем приобретения навыков
осмысления и расширения их содержания, навыков решения актуальных проблем
формирования общекультурных и профессиональных компетенций, научно"=исследовательской деятельности, подготовки к семинарам, лабораторным работам, сдаче зачетов и экзаменов. Самостоятельная работа студентов представляет собой совокупность
аудиторных и внеаудиторных занятий и работ.

Самостоятельная работа в рамках
образовательного процесса в вузе решает следующие задачи:
\begin{itemize}
\item закрепление и расширение знаний, умений, полученных студентами во время
аудиторных и внеаудиторных занятий, превращение их в стереотипы умственной и
физической деятельности;
\item приобретение дополнительных знаний и навыков по дисциплинам учебного плана;
\item формирование и развитие знаний и навыков, связанных с научно-исследовательской
деятельностью;
\item развитие ориентации и установки на качественное освоение образовательной
программы;
\item развитие навыков самоорганизации;
\item формирование самостоятельности мышления, способности к саморазвитию,
самосовершенствованию и самореализации;
\item выработка навыков эффективной самостоятельной профессиональной теоретической,
  практической и учебно-исследовательской деятельности.
\end{itemize}

\paragraph{Подготовка к лекции.} Качество освоения содержания конкретной дисциплины
прямо зависит от того, насколько студент сам, без внешнего принуждения формирует у
себя установку на получение на лекциях новых знаний, дополняющих уже имеющиеся по
данной дисциплине. Время на подготовку студентов к двухчасовой лекции по нормативам
составляет не менее 0,2 часа.

\paragraph{Подготовка к практическому занятию}
включает следующие элементы самостоятельной деятельности: четкое представление
цели и задач его проведения; выделение навыков умственной, аналитической, научной
деятельности, которые станут результатом предстоящей работы. Выработка навыков
осуществляется с помощью получения новой информации об изучаемых процессах и с
помощью знания о том, в какой степени в данное время студент владеет методами
исследовательской деятельности, которыми он станет пользоваться на практическом
занятии. Подготовка к практическому занятию нередко требует подбора материала,
данных и специальных источников, с которыми предстоит учебная работа. Студенты
должны дома подготовить к занятию 3--4 примера формулировки темы исследования,
представленного в монографиях, научных статьях, отчетах. Затем они самостоятельно
осуществляют поиск соответствующих источников, определяют актуальностьконкретного исследования процессов и явлений, выделяют основные способы
доказательства авторами научных работ ценности того, чем они занимаются. В ходе
самого практического занятия студенты сначала представляют найденные ими варианты
формулировки актуальности исследования, обсуждают их и обосновывают свое мнение о
наилучшем варианте. Время на подготовку к практическому занятию по нормативам
составляет не менее 0.2 часа.

\paragraph{Подготовка к семинарскому занятию.} Самостоятельная подготовка к семинару
направлена: на развитие способности к чтению научной и иной литературы; на поиск
дополнительной информации, позволяющей глубже разобраться в некоторых вопросах; на
выделение при работе с разными источниками необходимой информации, которая
требуется для полного ответа на вопросы плана семинарского занятия; на выработку
умения правильно выписывать высказывания авторов из имеющихся источников
информации, оформлять их по библиографическим нормам; на развитие умения
осуществлять анализ выбранных источников информации; на подготовку собственного
выступления по обсуждаемым вопросам; на формирование навыка оперативного
реагирования на разные мнения, которые могут возникать при обсуждении тех или иных
научных проблем. Время на подготовку к семинару по нормативам составляет не менее
0.2 часа.

\paragraph{Подготовка к коллоквиуму.} Коллоквиум представляет собой коллективное
обсуждение раздела дисциплины на основе самостоятельного изучения этого раздела
студентами. Подготовка к данному виду учебных занятий осуществляется в следующем
порядке. Преподаватель дает список вопросов, ответы на которые следует получить при
изучении определенного перечня научных источников. Студентам во внеаудиторное
время необходимо прочитать специальную литературу, выписать из нее ответы на
вопросы, которые будут обсуждаться на коллоквиуме, мысленно сформулировать свое
мнение по каждому из вопросов, которое они выскажут на занятии. Время на подготовку к
коллоквиуму по нормативам составляет не менее 0,2 часа.

\paragraph{Подготовка к контрольной работе.} Контрольная работа назначается после
изучения определенного раздела (разделов) дисциплины и представляет собой
совокупность развернутых письменных ответов студентов на вопросы, которые они
заранее получают от преподавателя. Самостоятельная подготовка к контрольной работе
включает в себя изучение конспектов лекций, раскрывающих материал, знание
которого проверяется контрольной работой; повторение учебного материала, полученного
при подготовке к семинарским, практическим занятиям и во время их проведения;
изучение дополнительной литературы, в которой конкретизируется содержание
проверяемых знаний; составление в мысленной форме ответов на поставленные в
контрольной работе вопросы; формирование психологической установки на успешное
выполнение всех заданий. Время на подготовку к контрольной работе по нормативам
составляет 2 часа.

\paragraph{Подготовка к зачету.} Самостоятельная подготовка к зачету должна
осуществляться в течение всего семестра. Подготовка включает следующие действия:
перечитать все лекции, а также материалы, которые готовились к семинарским и
практическим занятиям в течение семестра, соотнести эту информацию с вопросами,
которые даны к зачету, если информации недостаточно, ответы находят в предложенной
преподавателем литературе. Рекомендуется делать краткие записи. Время на подготовку к
зачету по нормативам составляет не менее 4 часов.

\paragraph{Подготовка к экзамену.} Самостоятельная подготовка к экзамену схожа с
подготовкой к зачету, особенно если он дифференцированный. Но объем учебного
материала, который нужно восстановить в памяти к экзамену, вновь осмыслить и понять,
значительно больше, поэтому требуется больше времени и умственных усилий. Важно
сформировать целостное представление о содержании ответа на каждый вопрос, что
предполагает знание разных научных трактовок сущности того или иного явления,процесса, умение раскрывать факторы, определяющие их противоречивость, знание имен
ученых, изучавших обсуждаемую проблему. Необходимо также привести информацию о
материалах эмпирических исследований, что указывает на всестороннюю подготовку
студента к экзамену. Время на подготовку к экзамену по нормативам составляет 36 часов
для бакалавров.

В ФБГОУ ВО «{{context.university.abbrev}}» организация самостоятельной работы студентов
регламентируется Положением о самостоятельной работе студентов, принятым Ученым
советом {{context.university.abbrev}} 22~июня~2012~г.

\section{Учебно"=методическое и информационное обеспечение дисциплины}
\noindent а)\ основная литература:
% \noindent а)\ основная литература (неограниченный доступ на сайте ЛЭТИ):
\begin{rdfctx}{\rdfsetctx{list}{syll wpdd:itemList !wpdd:BaseReferenceList !wpdd:ItemList}}
  \begin{enumerate}
      \item \rdf{list ^schema:member !wpdd:ListItem !wpdd:Reference}{...}
  \end{enumerate}
\end{rdfctx}

\noindent б)\ дополнительная литература:
% \noindent б)\ дополнительная литература (неограниченный доступ на сайте ЛЭТИ):
\begin{rdfctx}{\rdfsetctx{list}{syll wpdd:itemList !wpdd:AuxReferenceList !wpdd:ItemList}}
\begin{enumerate}
      \item \rdf{list ^schema:member !wpdd:ListItem !wpdd:Reference}{...}
\end{enumerate}
\end{rdfctx}

\noindent в)\ базы данных, информационно-справочные и поисковые системы:
\begin{rdfctx}{\rdfsetctx{list}{syll wpdd:itemList !wpdd:EReferenceList !wpdd:ItemList}}
\begin{enumerate}
  \item \rdf{list ^schema:member !wpdd:ListItem !wpdd:EReference}{Научная электронная библиотека «ELIBRARY.RU»\;:\;[сайт] -- URL: \url{http://elibrary.ru/defaultx.asp}}
  \item \rdf{list ^schema:member !wpdd:ListItem !wpdd:EReference}{Открытая электронная база ресурсов и исследований «Университетская информационная система РОССИЯ»\;:\;[сайт] -- URL: \url{http://uisrussia.msu.ru}}
  \item \rdf{list ^schema:member !wpdd:ListItem !wpdd:EReference}{Государственная информационная система «Национальная электронная библиотека»\;:\;[сайт] -- URL: \url{http://нэб.рф}}
\end{enumerate}
\end{rdfctx}
  В соответствии с п. 4.3.4. ФГОС ВО, обучающимся в течение всего периода обучения обеспечен неограниченный доступ (удаленный доступ) к электронно-библиотечным системам:
\begin{rdfctx}{\rdfsetctx{list}{syll wpdd:itemList !wpdd:EContractList !wpdd:ItemList}}
  \begin{enumerate}
  \item \rdf{list ^schema:member !wpdd:ListItem !wpdd:EReference}{ЭБС «Издательство Лань». ООО «Издательство Лань». Контракт № 92 от 12.11.2018 г. Акт от 14.11 2018 г.}
  \item \rdf{list ^schema:member !wpdd:ListItem !wpdd:EReference}{ЭБС ЭЧЗ «Библиотех». Государственный контракт № 019 от 22.02.2011 г. ООО «Библиотех». Лицензионное соглашение № 31 от 22.02.2011 г. Адрес доступа: \url{https://isu.bibliotech.ru/} Срок действия: с 22.11.2011 г. бессрочный.}
  \item \rdf{list ^schema:member !wpdd:ListItem !wpdd:EReference}{ЭБС «Национальный цифровой ресурс «Руконт». ЦКБ «Бибком». Контракт № 91 от 12.11.2018 г. Акт от 14.11.2018 г.}
  \item \rdf{list ^schema:member !wpdd:ListItem !wpdd:EReference}{ЭБС «Айбукс.ру/ibooks.ru». ООО «Айбукс». Контракт № 90 от 12.11.2018 г. Акт № 54 от 14.11.2018 г.}
  \item \rdf{list ^schema:member !wpdd:ListItem !wpdd:EReference rdfs:label}{Электронно-библиотечная система «ЭБС Юрайт». ООО «Электронное издательство Юрайт». Контракт № 70 от 04.10.2018 г.}
  \end{enumerate}
\end{rdfctx}
\noindent г)\ базы данных, информационно-справочные и поисковые системы:
\begin{rdfctx}{\rdfsetctx{list}{syll wpdd:itemList !wpdd:InternetResources !wpdd:ItemList}}
  \begin{enumerate}
  \item \rdf{list ^schema:member !wpdd:ListItem !wpdd:InternetResource}{....}
  \end{enumerate}
\end{rdfctx}

\section{Материально"=техническое обеспечение дисциплины}

\subsection{Учебно-лабораторное оборудование}

{{common.provision.lab}}

\subsection{Программное обеспечение}
\begin{rdfctx}{\rdfsetctx{list}{syll wpdd:itemList !wpdd:EReferenceList !wpdd:ItemList}}
  \begin{enumerate}
    \item \rdf{list ^schema:member !wpdd:ListItem !wpdd:Software}{Дистрибутив Linux 64 бит, }
    \item \rdf{list ^schema:member !wpdd:ListItem !wpdd:Software}{Дистрибутив Logtalk 3.30.0 и новее, }
    \item \rdf{list ^schema:member !wpdd:ListItem !wpdd:Software}{Редактор Libreoffice 7.0 или новее. }
  \end{enumerate}
\end{rdfctx}

{{common.provision.software}}

\section{Оценочные материалы для текущего контроля и промежуточной аттестации}


\subsection{Оценочные средства текущего контроля}
{{#rdfeach disc idd:hasCompetence}}
% {{this dcterms:identifier}}

% {{this rdfs:label}}

{{/rdfeach}}

\noindent{\footnotesize%
  \begin{tblr}{|X[3,l]|X[4,l]|X[2,c]|}
  \hline
  Вид контроля &
  Контролируемые темы & Контролируемые компетенции \\\hline
  {{#rdfeach syll wpdd:itemList ?wpdd:IndependentWorkTopicList ?wpdd:ItemList}}
  {{#with this ^schema:member ?wpdd:ListItem ?wpdd:IndependentWorkTopic}}
  & Тема~{{this dcterms:identifier}}.~{{this rdfs:label}} &
{{#rdfeach disc idd:hasCompetence}}
{{this dcterms:identifier}};
{{/rdfeach}}
  \\\hline
  {{/with}}
  {{/rdfeach}}
\end{tblr}}
\vspace{1em}

\paragraph{Примеры оценочных средств текущего контроля}
\vspace{1em}

\begin{rdfctx}{\rdfsetctx{list}{syll wpdd:itemList !wpdd:ExampleList !wpdd:CurrentAttestation !wpdd:ItemList}}
\rdf{list ^schema:member !wpdd:ListItem !wpdd:Example !wpdd:LaboratoryWork}{%
\emph{Лабораторная работа 1.} ...\par
\paragraph{Дано:} ....
\paragraph{Спроектировать:} ...
\paragraph{Дополнительные ограничения:} ...
}\\[1em]

\end{rdfctx}


\noindent Например:

\noindent Демонстрационный вариант контрольной работы №1 (№2, №3)

\noindent Демонстрационный вариант теста №1 (№2, №3)

\noindent Вопросы для собеседования №1 (№2, №3)

\noindent Вопросы для коллоквиума №1 (№2, №3)

\noindent Темы рефератов и др.

\subsection{Оценочные средства для промежуточной аттестации}

\paragraph{Список вопросов для промежуточной аттестации:}
\begin{rdfctx}{\rdfsetctx{list}{syll wpdd:itemList !wpdd:QuestionList !wpdd:ItemList}}
\begin{enumerate}
\item \rdf{list ^schema:member !wpdd:ListItem !wpdd:Question}{....}
\end{enumerate}
\end{rdfctx}
\paragraph{Примеры оценочных средств для промежуточной аттестации:}

% Примеры заданий находятся в разделе курса на сайте \url{https://edu.irnok.net}, а также в ЭИОС ИГУ, URL:~\url{educa.isu.ru}.

\begin{rdfctx}{\rdfsetctx{list}{syll wpdd:itemList !wpdd:ExampleList !wpdd:IntermediateAttestation !wpdd:ItemList}}
\begin{enumerate}
\item \rdf{list ^schema:member !wpdd:ListItem !wpdd:Example}{....}
\end{enumerate}
\end{rdfctx}
\vspace{1em}
\vfill
\paragraph{\normalfont Разработчик:}%
\rdf{syll wpdd:courseDC schema:author {foaf:name rdf:label} !foaf:Person }{....} % Иванов И.В., канд.техн.наук, доцент.
% \rdf{syll wpdd:courseDC schema:author {foaf:name rdf:label} !foaf:Person }{Черкашин~Е.А., к.т.н., доцент} % Иванов И.В., канд.техн.наук, доцент.

\end{document}

%%% Local Variables:
%%% mode: latex
%%% TeX-master: t
%%% End:
