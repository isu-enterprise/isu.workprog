% \documentclass[12pt]{extarticle}
\documentclass[12pt]{scrartcl}
\RequirePackage{tabularx}
\RequirePackage{indentfirst}
\RequirePackage{polyglossia}
\setmainlanguage{russian}
\setotherlanguage{english}
\setkeys{russian}{babelshorthands=true} % Переносы типа
                                        % красно"=зеленый
% \RequirePackage{minted}
%
\RequirePackage{luatextra}
\RequirePackage{unicode-math}
\defaultfontfeatures{Scale=MatchLowercase,Numbers=Lining,Ligatures=TeX}
\RequirePackage{microtype}
\SetProtrusion
    [name=std]
    {
      encoding={utf8},
      family=*}
    {
    « = {300,     },
    » = {    , 300},
    „ = {300,     },
    “ = {    , 300},
    ( = {300,     },
    ) = {    , 300},
    ! = {    , 300},
    ? = {    , 300},
    : = {    , 300},
    ; = {    , 300},
    . = {    , 300},
    - = {    , 300},
   {,}= {    , 300}
 }
 \microtypesetup{protrusion=true,expansion=true}

\setmainfont{Times New Roman}
\setsansfont{Fira Mono}
\setmonofont{Fira Code}[
     Scale=MatchLowercase,
     Numbers=SlashedZero,
     Ligatures=TeX,
     Numbers=Lining]
\newfontfamily{\cyrillicfonttt}{Fira Code}[
     Scale=MatchLowercase,
     Numbers=SlashedZero,
     Ligatures=TeX,
     Numbers=Lining]
\newfontfamily{\cyrillicfont}{Times New Roman}[
     Scale=MatchLowercase,
     Numbers=SlashedZero,
     Ligatures=TeX,
     Numbers=Lining]
\newfontfamily{\cyrillicfontsf}{Fira Mono}[
     Scale=MatchLowercase,
     Numbers=SlashedZero,
     Ligatures=TeX,
     Numbers=Lining]


     \RedeclareSectionCommand[%
     font=\large\rmfamily\bfseries
     ]{section}
     \RedeclareSectionCommand[%
     font=\rmfamily\bfseries
     ]{subsection}
     \RedeclareSectionCommand[%
     font=\normalsize\rmfamily\bfseries
     ]{subsubsection}
     \RedeclareSectionCommand[%
     font=\normalsize\rmfamily\bfseries
     ]{paragraph}
     \DeclareTOCStyleEntry{dottedtocline}{section}

\makeatletter
% \renewcommand\sectionlinesformat[4]{%
%   \@hangfrom{\hskip #2#3}{\MakeUppercase{#4}}%
% }
\makeatother
%\renewcommand{\sectioncatchphraseformat}[4]{%
%  \hskip #2#3\MakeUppercase{#4}%
%}

     \setlength{\parindent}{1cm}
     \setcounter{tocdepth}{\subsectiontocdepth}
\begin{document}
\begin{titlepage}
\begin{center}
  {\bfseries МИНИСТЕРСТВО ОБРАЗОВАНИЯ И НАУКИ РОССИЙСКОЙ ФЕДЕРАЦИИ}\par
  {\bfseries федеральное государственное бюджетное образовательное учреждение}\par
  {\bfseries \MakeUppercase{ <<{{context.university rdfs:label}}>>} }\par
{ФГБОУ ВО «{{context.university.abbrev}}»}\par
\MakeUppercase{ {{context.institute rdfs:label}} }\par
\vspace{1ex}
{\bfseries {{disc idd:chair rdfs:label}}}
\end{center}
\vspace{2em}
\begin{tabularx}{\textwidth}{XX}
  & <<УТВЕРЖДАЮ>> \\
  & {{context.institute.position}} {{context.institute.abbrev}} {{context.university.abbrev}} \\
  & \underline{\hspace{3cm}}\;{{context.director foaf:name}}
\end{tabularx}
\vfil
\begin{center}\large
  {\bfseries Рабочая программа дисциплины (модуля) }\\[1ex]
  {\bfseries {{disc dcterms:identifier}}~{{disc idd:discipline rdfs:label}}}
\end{center}
\vfil
\begin{tabularx}{\textwidth}{XXX}
  {\bfseries Направление подготовки} & {\hyphenpenalty=100000 {{curr idd:specialty dcterms:identifier}}~{{context.specialty rdfs:label}}}\\
  {\bfseries Направленность (профиль) подготовки} & {{context.profile rdfs:label}}\\
  {\bfseries Квалификация выпускника} & {{context.level rdfs:label}}\\
  {\bfseries Форма обучения} & {{context.mural rdfs:label}}
\end{tabularx}
\vfil
\vfil
\begin{center}
  {{context.city}}\;--\;{{context.enrolledIn}}
\end{center}
\end{titlepage}
\newpage
\tableofcontents
\newpage
\section{Цели и задачи дисциплины}
\paragraph{Цели:}\\
\par
\paragraph{Задачи:}\\

\section{Место дисциплины в структуре ОПОП ВО}
Учебная дисциплина {{disc dcterms:identifier}}~<<{{disc idd:discipline rdfs:label}}>> относится к
Блоку~{{disc idd:block dcterms:identifier}} образовательной программы, {{disc idd:blockPart rdfs:comment}}.

Для изучения данной учебной дисциплины необходимы знания, умения и навыки,
формируемые предшествующими дисциплинами: ...

Перечень последующих учебных дисциплин, для которых необходимы знания, умения и
навыки, формируемые данной учебной дисциплиной: ....


\section{Требования к результатам освоения дисциплины}

Процесс освоения дисциплины направлен на формирование следующих компетенций в
соответствии с ФГОС ВО и ОП ВО по направлению подготовки {{context.specialty dcterms:identifier}}~{{context.specialty rdfs:label}}:

/ОПК-1 Способен применять фундаментальные знания, полученные в области
математических и (или) естественных наук, и использовать их в профессиональной
деятельности./

В результате освоения дисциплины обучающийся должен

знать: ...;

уметь: ...;

владеть: ....

\section{Содержание и структура дисциплины}

Объем дисциплины составляет {{disc idd:credits idd:expert}} зачетных ед., {{disc idd:hours idd:expert}} час.

Форма промежуточной аттестации: {{disc idd:controlType rdfs:label}}.

\subsection{Содержание дисциплины, структурированное по темам, c указанием видов
  учебных занятий и отведенного на них количества академических часов}

\subsection{План внеаудиторной самостоятельной работы обучающихся по дисциплине}

\subsection{Содержание учебного материала}

\subsubsection{Перечень семинарских, практических занятий и лабораторных работ}

\subsubsection{Перечень тем (вопросов), выносимых на самостоятельное изучение студентами
в рамках самостоятельной работы}

\subsection{Методические указания по организации самостоятельной работы студентов}

Самостоятельная работа студентов всех форм и видов обучения является одним из
обязательных видов образовательной деятельности, обеспечивающей реализацию
требований Федеральных государственных стандартов высшего образования. Согласно
требованиям нормативных документов самостоятельная работа студентов является
обязательным компонентом образовательного процесса, так как она обеспечивает
закрепление получаемых на лекционных занятиях знаний путем приобретения навыков
осмысления и расширения их содержания, навыков решения актуальных проблем
формирования общекультурных и профессиональных компетенций, научно"=исследовательской деятельности, подготовки к семинарам, лабораторным работам, сдаче зачетов и экзаменов. Самостоятельная работа студентов представляет собой совокупность
аудиторных и внеаудиторных занятий и работ.

Самостоятельная работа в рамках
образовательного процесса в вузе решает следующие задачи:
\begin{itemize}
\item закрепление и расширение знаний, умений, полученных студентами во время
аудиторных и внеаудиторных занятий, превращение их в стереотипы умственной и
физической деятельности;
\item приобретение дополнительных знаний и навыков по дисциплинам учебного плана;
\item формирование и развитие знаний и навыков, связанных с научно-исследовательской
деятельностью;
\item развитие ориентации и установки на качественное освоение образовательной
программы;
\item развитие навыков самоорганизации;
\item формирование самостоятельности мышления, способности к саморазвитию,
самосовершенствованию и самореализации;
\item выработка навыков эффективной самостоятельной профессиональной теоретической,
  практической и учебно-исследовательской деятельности.
\end{itemize}

\paragraph{Подготовка к лекции.} Качество освоения содержания конкретной дисциплины
прямо зависит от того, насколько студент сам, без внешнего принуждения формирует у
себя установку на получение на лекциях новых знаний, дополняющих уже имеющиеся по
данной дисциплине. Время на подготовку студентов к двухчасовой лекции по нормативам
составляет не менее 0,2 часа.

\paragraph{Подготовка к практическому занятию}
включает следующие элементы самостоятельной деятельности: четкое представление
цели и задач его проведения; выделение навыков умственной, аналитической, научной
деятельности, которые станут результатом предстоящей работы. Выработка навыков
осуществляется с помощью получения новой информации об изучаемых процессах и с
помощью знания о том, в какой степени в данное время студент владеет методами
исследовательской деятельности, которыми он станет пользоваться на практическом
занятии. Подготовка к практическому занятию нередко требует подбора материала,
данных и специальных источников, с которыми предстоит учебная работа. Студенты
должны дома подготовить к занятию 3--4 примера формулировки темы исследования,
представленного в монографиях, научных статьях, отчетах. Затем они самостоятельно
осуществляют поиск соответствующих источников, определяют актуальностьконкретного исследования процессов и явлений, выделяют основные способы
доказательства авторами научных работ ценности того, чем они занимаются. В ходе
самого практического занятия студенты сначала представляют найденные ими варианты
формулировки актуальности исследования, обсуждают их и обосновывают свое мнение о
наилучшем варианте. Время на подготовку к практическому занятию по нормативам
составляет не менее 0.2 часа.

\paragraph{Подготовка к семинарскому занятию.} Самостоятельная подготовка к семинару
направлена: на развитие способности к чтению научной и иной литературы; на поиск
дополнительной информации, позволяющей глубже разобраться в некоторых вопросах; на
выделение при работе с разными источниками необходимой информации, которая
требуется для полного ответа на вопросы плана семинарского занятия; на выработку
умения правильно выписывать высказывания авторов из имеющихся источников
информации, оформлять их по библиографическим нормам; на развитие умения
осуществлять анализ выбранных источников информации; на подготовку собственного
выступления по обсуждаемым вопросам; на формирование навыка оперативного
реагирования на разные мнения, которые могут возникать при обсуждении тех или иных
научных проблем. Время на подготовку к семинару по нормативам составляет не менее
0.2 часа.

\paragraph{Подготовка к коллоквиуму.} Коллоквиум представляет собой коллективное
обсуждение раздела дисциплины на основе самостоятельного изучения этого раздела
студентами. Подготовка к данному виду учебных занятий осуществляется в следующем
порядке. Преподаватель дает список вопросов, ответы на которые следует получить при
изучении определенного перечня научных источников. Студентам во внеаудиторное
время необходимо прочитать специальную литературу, выписать из нее ответы на
вопросы, которые будут обсуждаться на коллоквиуме, мысленно сформулировать свое
мнение по каждому из вопросов, которое они выскажут на занятии. Время на подготовку к
коллоквиуму по нормативам составляет не менее 0,2 часа.

\paragraph{Подготовка к контрольной работе.} Контрольная работа назначается после
изучения определенного раздела (разделов) дисциплины и представляет собой
совокупность развернутых письменных ответов студентов на вопросы, которые они
заранее получают от преподавателя. Самостоятельная подготовка к контрольной работе
включает в себя изучение конспектов лекций, раскрывающих материал, знание
которого проверяется контрольной работой; повторение учебного материала, полученного
при подготовке к семинарским, практическим занятиям и во время их проведения;
изучение дополнительной литературы, в которой конкретизируется содержание
проверяемых знаний; составление в мысленной форме ответов на поставленные в
контрольной работе вопросы; формирование психологической установки на успешное
выполнение всех заданий. Время на подготовку к контрольной работе по нормативам
составляет 2 часа.

\paragraph{Подготовка к зачету.} Самостоятельная подготовка к зачету должна
осуществляться в течение всего семестра. Подготовка включает следующие действия:
перечитать все лекции, а также материалы, которые готовились к семинарским и
практическим занятиям в течение семестра, соотнести эту информацию с вопросами,
которые даны к зачету, если информации недостаточно, ответы находят в предложенной
преподавателем литературе. Рекомендуется делать краткие записи. Время на подготовку к
зачету по нормативам составляет не менее 4 часов.

\paragraph{Подготовка к экзамену.} Самостоятельная подготовка к экзамену схожа с
подготовкой к зачету, особенно если он дифференцированный. Но объем учебного
материала, который нужно восстановить в памяти к экзамену, вновь осмыслить и понять,
значительно больше, поэтому требуется больше времени и умственных усилий. Важно
сформировать целостное представление о содержании ответа на каждый вопрос, что
предполагает знание разных научных трактовок сущности того или иного явления,процесса, умение раскрывать факторы, определяющие их противоречивость, знание имен
ученых, изучавших обсуждаемую проблему. Необходимо также привести информацию о
материалах эмпирических исследований, что указывает на всестороннюю подготовку
студента к экзамену. Время на подготовку к экзамену по нормативам составляет 36 часов
для бакалавров.

В ФБГОУ ВО «{{context.university.abbrev}}» организация самостоятельной работы студентов
регламентируется Положением о самостоятельной работе студентов, принятым Ученым
советом {{context.university.abbrev}} 22~июня~2012~г.

\section{Учебно"=методическое и информационное обеспечение дисциплины}
\noindent а)\ основная литература:

\noindent б)\ дополнительная литература:

\noindent в)\ базы данных, информационно-справочные и поисковые системы:

\section{Материально"=техническое обеспечение дисциплины}

\subsection{Учебно-лабораторное оборудование}

ЭТОТ РАЗДЕЛ НЕ ЗАПОЛНЯТЬ

\subsection{Программное обеспечение}

ПЕРЕЧИСЛИТЬ ПРОГРАММНОЕ ОБЕСПЕЧЕНИЕ ТРЕБУЕМОЕ ДЛЯ ДИСЦИПЛИНЫ

\section{Оценочные материалы для текущего контроля и
промежуточной аттестации}


\subsection{Оценочные средства текущего контроля}

Вид контроля

Контролируемые темы

Примеры оценочных средств текущего контроля

\subsection{Оценочные средства для промежуточной аттестации}

\paragraph{Список вопросов для промежуточной аттестации:}

\paragraph{Примеры оценочных средств для промежуточной аттестации:}

\par
\paragraph{\normalfont Разработчик:} {{context.developer}} %Головко Е. А., кандидат физ.-мат. наук, доцент

\end{document}

%%% Local Variables:
%%% mode: latex
%%% TeX-master: t
%%% End:
